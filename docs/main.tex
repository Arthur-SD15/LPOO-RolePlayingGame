\documentclass[11pt]{uiobrev}
\usepackage[utf8]{inputenc}
\usepackage[T1]{fontenc}
\usepackage[portuguese]{babel}
\usepackage{tgtermes, xurl, textpos, graphicx, geometry}
\usepackage{enumitem}
\usepackage{fancyhdr}

\geometry{a4paper, vmargin=3.5cm, hmargin=2.5cm}
\name{Arthur Dantas}
\date{17 de Novembro de 2024}


\pagestyle{fancy}
\fancyhf{} 
\fancyhead[R]{\thepage}
\renewcommand{\headrulewidth}{0pt}

\begin{document}

\begin{letter}{\\Arthur Silva Dantas \\ Graziela Santos de Araújo}

    \opening{Assunto: Trabalho Prático - Linguagem de Programação Orientada a Objetos}

    Prezados,

    Este relatório destina-se ao trabalho prático da disciplina de Linguagem de Programação Orientada a Objetos, onde o projeto simula um jogo de RPG utilizando os princípios de Orientação a Objetos e aplicando o conteúdo abordado ao longo da disciplina.

    \bigskip
    Para obter mais informações sobre o código completo e consultar uma parte adicional da documentação, o projeto está disponível no GitHub. 
    \\ Disponível em: \url{https://github.com/Arthur-SD15/LPOO-RolePlayingGame}.

    \closing{Atenciosamente,}
\thispagestyle{empty} 
\end{letter}

\newpage

\begin{center}
    \Large \textbf{Sumário}
\end{center}
\vspace{1cm}

\begin{enumerate}
    \item \textbf{Projeto Geral} \dotfill \textbf{1}
    \item \textbf{Organização Geral} \dotfill \textbf{1}
    \item \textbf{Organização Detalhada} \dotfill \textbf{1}
        \begin{enumerate}[label*=\arabic*.] 
            \item \textbf{Personagem.java} \dotfill \textbf{1}
            \item \textbf{Arma.java} \dotfill \textbf{1}
            \item \textbf{Mago.java} \dotfill \textbf{1}
            \item \textbf{Paladino.java} \dotfill \textbf{1}
            \item \textbf{Clerigo.java} \dotfill \textbf{1}
            \item \textbf{ArmaMago.java} \dotfill \textbf{1}
            \item \textbf{ArmaPaladino.java} \dotfill \textbf{1}
            \item \textbf{ArmaClerigo.java} \dotfill \textbf{1}
            \item \textbf{Transmutação.java} \dotfill \textbf{1}
            \item \textbf{Psikappa.java} \dotfill \textbf{1}
            \item \textbf{Espada.java} \dotfill \textbf{1}
            \item \textbf{Lança.java} \dotfill \textbf{1}
            \item \textbf{Martelo.java} \dotfill \textbf{1}
            \item \textbf{Maça.java} \dotfill \textbf{1}
        \end{enumerate}
    \item \textbf{Dificuldades Encontradas} \dotfill \textbf{1}
    \item \textbf{Soluções} \dotfill \textbf{1}
\end{enumerate}
\thispagestyle{empty}

\newpage

\setcounter{page}{1}

\noindent
\Large \textbf{1. Introdução}

Lorem ipsum

\end{document}